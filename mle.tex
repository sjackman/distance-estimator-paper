\documentclass[letterpaper,12pt]{article}
\usepackage{amsmath}
\usepackage[utf8]{inputenc}
\DeclareMathOperator*{\argmax}{arg\,max}
\DeclareMathOperator{\MLE}{MLE}

\title{Estimating the distance between two sequences using paired-end reads}
\author{Shaun D Jackman, İnanç Birol \\ sjackman@bcgsc.ca \\
Canada's Michael Smith Genome Sciences Centre \\
Vancouver, British Columbia, V5Z 4E6, Canada
}

\begin{document}
\maketitle

\begin{abstract}
Paired-end reads may be used to estimate the distance between two
sequences. Comparing a statistic, such as the mean, of the sample
population of fragment sizes to the global population of fragment
sizes is a trivial but flawed estimator. The maximum likelihood
estimator yields more accurate estimates.
\end{abstract}

\section*{Background}

The goal is to estimate the size of the gap between two sequences.

\section*{Results}

To estimate the distance between two sequences, we start by mapping
paired-end reads to the two sequences, which are then ordered and
oriented to agree with the orientation of the reads. We establish a
coordinate system where $-l_1$ and $-1$ are the first and last base of
the first sequence of length $l_1$, and $0$ and $l_2-1$ are the first
and last base of the second sequence of length $l_2$.  An observed
fragment size, $x_i$, is calculated for each pair by calculating the
difference of the mapped position of the first sequenced base of each
of the two reads. This observed fragment size differs from the actual
fragment size by the size of the gap between the two sequences,
$\theta_0$.

The final input is the distribution of fragment sizes of the library,
which is derived empirically by mapping the reads to a reference
sequence or sequences assembled \textit{de novo} and determining the
inferred fragment size distribution.

\subsection*{Estimator using the mean}
At first glance, this task appears to be rather simple, and a simple
solution presents itself readily. A reasonable estimate, $\hat
\theta_\text{mean}$, of the size of the gap is the difference between
the mean of the population, $\mu$, and the mean of the sample,
$\bar x$.

\begin{equation*}
\hat \theta_\text{mean} = \mu - \bar x
\end{equation*}

\subsection*{Maximum likelihood estimator}

The estimate of the distance between the two sequences can be improved
by using the probability distribution in its entirety rather than a
summary statistic. Let the probability of observing a fragment of size
$x$ selected at random from the population be $f_X(x)$, and the
probability of observing a fragment of size $x$ that spans a gap of
size $\theta$ be $f_\theta(x)$. With a sample of $n$ observed fragment
sizes, $x_1, \dotsc , x_n$, the likelihood that the two sequences are
separated by a distance of $\theta$ bases is $\mathcal{L}(\theta \mid
x_1, \dotsc , x_n)$.

The most likely estimate of the size of the gap between the two
sequences is the value $\hat \theta_{\MLE}$ that maximizes the
likelihood function, or conveniently, the log likeliehood function,
since the log function is a monotonic transformation.

\begin{align*}
\hat \theta_{\MLE}
&= \argmax_\theta \mathcal{L}(\theta \mid x_1, \dotsc, x_n)
	= \argmax_\theta \prod_{i=1}^n f_\theta(x_i) \\
&= \argmax_\theta \log \mathcal{L}(\theta \mid x_1, \dotsc, x_n)
	= \argmax_\theta \sum_{i=1}^n \log f_\theta(x_i)
\end{align*}

\subsection*{Distribution of fragment sizes that span the gap}

The distribution of observed fragment sizes that span the gap is equal
to the population distribution, $P(X=x)$, shifted by the size of the
gap, $\theta$. Since we can only observe fragments that actually span
the gap, we use Bayes' thereom to determine the conditional
probability of observing a fragment of size $x$ given that it spans
the gap of size $\theta$.

\begin{align*}
f_{\theta}(x)
&= P(X=x+\theta \mid \text{fragment spans gap}) \\
&= \frac{ P(\text{fragment spans gap} \mid X=x+\theta) P(X=x+\theta) }
{ P(\text{fragment spans gap}) } \\
&\propto P(\text{fragment spans gap} \mid X=x+\theta) P(X=x+\theta)
\end{align*}

Assume the reads are sampled uniformly from the genome between the
coordinates $a$ and $b$, where $b - a$ is the size of the genome.

\begin{equation*}
P(U=u) = \begin{cases}
\frac{1}{b-a} & a \leq u < b \\
0 & \text{otherwise}
\end{cases}
\end{equation*}

The probability that a fragment of size $X$ spans the gap is the
probability that the fragment's left coordinate, $U$, falls to the
left of the gap, and its right coordinate, $U+X$, falls to the right
of gap.

\begin{align*}
\text{Let}\; w_\theta(x + \theta)
&=P(\text{fragment spans gap} \mid X=x+\theta) \\
&=P(-l_1 \leq U < 0 \wedge \theta \leq U + X < l_2 + \theta \mid X=x+\theta) \\
&= P(-l_1 \leq U < 0 \wedge \theta \leq U + x + \theta < l_2 + \theta) \\
&= P(-l_1 \leq U < 0 \wedge 0 \leq U + x < l_2) \\
&= w(x)
\end{align*}

This shows that $w_\theta(x+\theta)$, the probability that a fragment
of size $x+\theta$ spans the gap, is independent of the size of the
gap, $\theta$, and depends only on the observed size of the fragment,
$x$.

\begin{align*}
w(x)
&= P(-l_1 \leq U < 0 \wedge 0 \leq U + x < l_2) \\
&= P(-l_1 \leq U < 0 \wedge -x \leq U < l_2 - x) \\
&= P(\max(-l_1, -x) \leq U < \min(0, l_2 - x)) \\
&= \sum_{i=\max(-l_1, -x)}^{\min(0, l_2 - x) - 1} P(U=i) \\
&= \sum_{i=\max(-l_1, -x)}^{\min(0, l_2 - x) - 1} \frac{1}{b-a} \\
&= \max(0, \min(0, l_2 - x) - \max(-l_1, -x)) \frac{1}{b-a} \\
&\propto \max(0, \min(0, l_2 - x) - \max(-l_1, -x)) \\
&= \max(0, \min(0, l_2 - x) + \min(l_1, x)) \\
&= \max(0, \min(x, l_1, l_2, l_1 + l_2 - x))
\end{align*}

Without loss of generality, assume $l_1 \leq l_2$.
\begin{equation*}
w(x) \propto \begin{cases}
x & 0 \leq x < l_1 \\
l_1 & l_1 \leq x < l_2 \\
l_1 + l_2 - x & l_2 \leq x < l_1 + l_2 \\
0 & \text{otherwise}
\end{cases}
\end{equation*}

\begin{equation*}
f_\theta(x) \propto f_X(x + \theta) w(x)
\end{equation*}

\begin{equation*}
f_\theta(x) = \frac{ f_X(x + \theta) w(x) }
	{ \sum_{j=1}^\infty f_X(j + \theta) w(j) }
\end{equation*}

\subsection*{Solving the maximum likelihood estimator}

We now substitute the distribution of observed fragment sizes,
$f_\theta(x)$, into the formula for the maximum likelihood estimator.

\begin{align*}
\mathcal{L}(\theta \mid x_1, \dotsc, x_n)
&= \prod_{i=1}^n f_\theta(x_i) \\
&= \prod_{i=1}^n \frac{ f_X(x_i + \theta) w(i) }
	{ \sum_{j=1}^\infty f_X(j + \theta) w(j) } \\
&= \frac{ \prod_{i=1}^n f_X(x_i + \theta) w(i) }
	{ \left( \sum_{j=1}^\infty f_X(j + \theta) w(j) \right) ^n }
\end{align*}

\begin{equation*}
\log \mathcal{L}(\theta \mid x_1, \dotsc, x_n)
= \sum_{i=1}^n \log f_X(x_i + \theta)
	+ \sum_{i=1}^n \log w(i)
	- n \log \sum_{j=1}^\infty f_X(j + \theta) w(j)
\end{equation*}

\begin{align*}
\hat \theta_{\MLE}
&= \argmax_\theta \log \mathcal{L}(\theta \mid x_1, \dotsc, x_n) \\
&= \argmax_\theta \left[ \sum_{i=1}^n \log f_X(x_i + \theta)
	+ \sum_{i=1}^n \log w(i)
	- n \log \sum_{j=1}^\infty f_X(j + \theta) w(j) \right] \\
&= \argmax_\theta \left[ \sum_{i=1}^n \log f_X(x_i + \theta)
	- n \log \sum_{j=1}^\infty f_X(j + \theta) w(j) \right]
\end{align*}

Finding the value of $\theta$ that maximizes the likelihood function
is an optimization problem. When the range of possible values of
$\theta$ is small, that is when the fragment size of the sequencing
library is small, it is reasonable to calculate exhaustively every
value of $\mathcal{L}(\theta)$ to find the maximum.

\section*{Conclusion}

This distance estimation algorithm is implemented by the
\textit{ABySS} assembly software in the utility \textit{DistanceEst},
which requires as its input the distribution of fragment sizes of the
sequencing library and a \textit{SAM}-formatted file of paired-end
reads that map to different sequences.

\section*{Acknowledgements}

Jared Simpson implemented a maximum likelihood estimator for
estimating distances between sequences in the first release of the
software \textit{ABySS}.

\section*{References}

ABySS: A parallel assembler for short read sequence data. Simpson JT,
Wong K, Jackman SD, Schein JE, Jones SJ, Birol I. Genome Research,
2009-June.

\end{document}
